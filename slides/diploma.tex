\documentclass{beamer}

\usetheme{Warsaw}

\usepackage[T1,T2A]{fontenc}
\usepackage[utf8]{inputenc}
\usepackage[english,russian]{babel}
\usepackage{amssymb,amsfonts,amsmath,mathtext,float}  

\makeatletter
\renewcommand{\@biblabel}[1]{#1.} % Заменяем библиографию с квадратных скобок на точку:
\makeatother

\usepackage{csquotes}

%\usepackage{bibunits}  
\setbeamertemplate{bibliography item}[text]
%\defaultbibliography{plain}
%\defaultbibliographystyle{IEEEtran}

\usepackage{caption}
\DeclareCaptionLabelSeparator{par}{\par}
\newcommand\tablecaption[1]{
	\captionsetup{labelsep=par,justification=raggedleft}
	\caption{#1}
}

\usepackage{bm}

\usepackage{multirow}

\usepackage{rotating}

\usepackage[noend]{algorithmic}
\usepackage{algorithm}

\newcommand{\algname}[1]{\textsc{#1}}

\renewcommand{\algorithmicrequire}{\textbf{Input:}}
\renewcommand{\algorithmicensure}{\textbf{Output:}}
\renewcommand{\algorithmicprint}{\textbf{output}}
\renewcommand{\algorithmiccomment}[1]{\hspace*{\fill}\{#1\}}

\algsetup{indent=2em}

\usepackage{amsthm}
\theoremstyle{definition}
%\newtheorem{definition}{Определение}

\addto\captionsrussian{\renewcommand{\bibname}{Список использованных источников}}


\def\cp#1{\preccurlyeq_{#1}}

\def\PP{\mathbb{P}}
\def\UCPP{\context^\PP_{\sim}}
\def\mpref{\leq}
\def\cprime{^{\sim}}

\def\ABC{A \cp{C} B}
\def\DEF{D \cp{F} E}

\def\minivan{\textrm{minivan}}
\def\suv{\textrm{SUV}}
\def\red{\textrm{red}}
\def\white{\textrm{white}}
\def\bright{\textrm{bright}}
\def\price{\textrm{price}}
\def\dark{\textrm{dark}}

\def\Z{\mathcal{Z}}

\makeatletter
\def\@makechapterhead#1{%
	\vspace*{50\p@}%
	{\parindent \z@ \raggedright \normalfont
		\interlinepenalty\@M
		\Huge\bfseries  \thechapter.\enskip #1\par\nobreak
		\vskip 40\p@
	}}
\makeatother
	

\makeatletter
\renewcommand{\ALG@name}{Алгоритм}
\renewcommand{\listalgorithmname}{Список алгоритмов}
\makeatother

%\usepackage[unicode]{hyperref}

\beamertemplatenavigationsymbolsempty
\setbeamertemplate{footline}[frame number]
\setbeamertemplate{itemize items}[default]
\setbeamertemplate{enumerate items}[default]

\title[Выявление предпочтений \enquote{при прочих равных}] % (optional, only for long titles)
{Экспериментальное исследование и модификация алгоритма выявления предпочтений \enquote{при прочих равных}}
\author[Садыков~Э.Р.]{Студент: Садыков~Эрнест~Рашидович, 401ПИ \\ 
	\vspace{1em}
	Научный руководитель: Объедков Сергей Александрович, к.т.н., доцент департамента анализа данных и искусственного интеллекта факультета компьютерных наук
}
\institute
{
	Национальный исследовательский университет\\
	Высшая школа экономики
}
\date{Москва, 2015}

\begin{document}
	\frame{\titlepage}
	\begin{frame}
		\frametitle{Актуальность проблемы}
		Применения рекомендательных алгоритмов:
		\begin{itemize}
			\item Интернет-магазины (amazon, ozon)
			\item Социальные сети (reddit, digg)
			\item Специализированные рекомендательные системы (last.fm, Имхонет)
		\end{itemize}
		
		\pause
		
		\vspace{1.5em}
		Рекомендации удерживают внимание пользователя на сайте

		\pause 
		
		\vspace{1.5em}
		В основе рекомендательных систем – алгоритмы выявления предпочтений
	\end{frame}
	
	\begin{frame}
		\frametitle{Цель и задачи}
		\emph{Цель}: реализовать и протестировать алгоритм обучения предпочтениям «при прочих равных» для проверки его применимости к реальным данным; разработать и протестировать модификации алгоритма, улучшающие его характеристики

		\pause
		\vspace{1em}
		
		\emph{Задачи}:
		\begin{itemize}
			\item Реализовать базовый алгоритм, описанный в \cite{Obiedkov:2013}
			\item Провести апробацию алгоритма на реальных данных 
			\item Провести сравнение работы базового алгоритма с другими методами машинного обучения
			\item Разработать модификации алгоритма, повышающие качество работы алгоритма
			\item Сравнить результаты работы модификаций алгоритма с другими методами машинного обучения
		\end{itemize}
	\end{frame}
	
	\begin{frame}
		\frametitle{Пример контекста}
		\begin{center}
		\includegraphics[width=80mm]{./images/context.png}
		\end{center}
		\vspace{-1.5ex}
		 $c_1, \dotsc, c_5$ – объекты (в данном случае автомобили) \\
		 ``миниван'', ``внедорожник'', \dots, ``темный интерьер'' – признаки \\
		 \vspace{1ex}
		 справа – граф предпочтений: $c_5$ предпочтительнее $c_1$; $c_2$ и $c_3$ несравнимы
	\end{frame}
	
	\begin{frame}
		Интуитивное описание метода 
	\end{frame}
	
	\begin{frame}
		\frametitle{Наборы данных}
		\begin{enumerate}
			\item \emph{Набор данных о предпочтениях в автомобилях}: 60 человек указывали свои предпочтения, выбирая из 10 автомобилей \cite{dataset:Abbasnejad:2013} \\
			\url{http://users.cecs.anu.edu.au/~u4940058/CarPreferences.html}
			\vspace{1em}
			\item \emph{Набор данных о предпочтениях в суши}: 5000 человек ранжировали 10 суши \cite{Kamishima:2003}
			\url{http://www.kamishima.net/sushi/}
		\end{enumerate}
	\end{frame}
	
	\begin{frame}
		%\frametitle{Проведение экспериментов}
		проведение экспериментов
	\end{frame}
	
	\begin{frame}
		\frametitle{Первый эксперимент}
		\framesubtitle{над базовой версией алгоритма}
		\begin{center}
			\begin{tabular}{|l|ll|}
				\hline
				Алгоритм             & 1-out & 2-out \\ \hline
				\enquote{при прочих равных}      & $59.09 \pm 3.08$ & $52.94 \pm 0.97$ \\
				C4.5                 & $79.91 \pm 7.29$ & $71.61 \pm 8.97$ \\
				Наивный байес. клас. & $83.76 \pm 7.2$ & $82.69 \pm 9.68$ \\ \hline
			\end{tabular}
		\end{center}
		
		\vspace{1.5em}
		Алгоритм выявления предпочтений \enquote{при прочих равных} для большинства пар $(x,y)$ находит аргументы и за $x \leq y$, и за $y \leq x$.
	\end{frame}
	
	\begin{frame}
		\frametitle{Пример}
		\begin{center}
			Сравним два автомобиля:
		\end{center}
		\begin{columns}[c] 
			\column{.5\textwidth} 
			Красный внедорожник с темным интерьером
	    	\column{.5\textwidth}
	    	Красный миниван со светлым интерьером
		\end{columns}
		\begin{center}
			\vspace{2em}
			\emph{лучше} потому что \\внедорожники лучше миниванов \\
			\vspace{2em}
			\emph{хуже} потому что \\темный интерьер хуже светлого 
		\end{center}
	\end{frame}
	
	\begin{frame}
		\frametitle{Модификация 1}
		\framesubtitle{подсчет ``подтверждающих'' пар объектов}
		Считаем количество пар объектов обучающей выборки, которые ``подтверждают'' предпочтение. 
		\vspace{1.4em}
		\begin{columns}[c] 
			\column{.5\textwidth} 
			Красный внедорожник с темным интерьером
			\column{.5\textwidth}
			Красный миниван со светлым интерьером
		\end{columns}
		\begin{center}
			\vspace{0.5em}
			\emph{лучше}: \\ 
			2 пары ``внедорожники лучше миниванов'' \\
			2 пары ``внедорожники лучше светлого интерьера при одинаковом экстерьере'' \\
			\vspace{1em}
			\emph{хуже}: \\
			3 пары ``темный интерьер хуже светлого''
		\end{center}
		
		\vspace{1.5em}
		Таким образом, красный внедорожник с темным интерьером {\color{green} лучше} красного минивана со светлым интерьером (4 против 3).
	\end{frame}
	
	\begin{frame}
		\frametitle{Модификация 2}
		\framesubtitle{поиск наиболе ``сильных'' предпочтений}
		Ищем такое предпочтение, которое подкреплено максимальным количеством ``подтверждающих'' пар.
		\vspace{1.4em}
		\begin{columns}[c] 
			\column{.5\textwidth} 
			Красный внедорожник с темным интерьером
			\column{.5\textwidth}
			Красный миниван со светлым интерьером
		\end{columns}
		\begin{center}
			\vspace{0.5em}
			\emph{лучше}: \\ 
			2 пары ``внедорожники лучше миниванов'' \\
			2 пары ``внедорожники лучше светлого интерьера при одинаковом экстерьере'' \\
			\vspace{1em}
			\emph{хуже}: \\
			3 пары ``темный интерьер хуже светлого''
		\end{center}
		
		\vspace{1.5em}
		Таким образом, красный внедорожник с темным интерьером {\color{red} хуже} красного минивана со светлым интерьером (2 против 3).
	\end{frame}
	
	\begin{frame}
		Модификация: бинарные отношения	
	\end{frame}
	
	
	\begin{frame}
		\frametitle{Результаты}
		\begin{itemize}
			\item Реализован базовый алгоритм выявления предпочтений «при прочих равных»
			\item Проведена апробация алгоритма на двух реальных наборах данных: cars dataset \cite{dataset:Abbasnejad:2013}, sushi dataset \cite{Kamishima:2003} 
			\item Проведено сравнение работы базового алгоритма с другими методами машинного обучения
			\item Разработано 3 модификации алгоритма, повышающие качество работы базового алгоритма (подумать)
			\item Экспериментально обоснована состоятельность подхода, лежащего в основе базовой версии алгоритма и показана  эффективность его (алгоритма) модификаций
		\end{itemize}
		\pause
		\begin{itemize}
			\item {\color{gray} Написан инструментарий для сравнения алгоритмов: классы для проведения экспериментов, утилита с GUI для запуска экспериментов}
		\end{itemize}
	\end{frame}
		
	\begin{frame}[plain,c]
		\begin{center}
			\Huge Спасибо за внимание!
		\end{center}
	\end{frame}

	\begin{frame}[allowframebreaks]
		\frametitle{Список использованных источников}
		\bibliographystyle{apalike}
		\bibliography{bibliography}
	\end{frame}
	
	%%\setcounter{page}{2}
	%\tableofcontents 
	%\chapter*{Реферат}
\addcontentsline{toc}{chapter}{Реферат}

Задача ранжирования одна из наиболее активно развиваемых областей обучения предпочтениям. В этой работе представлена реализация алгоритма выявления предпочтений при прочих равных. Этот алгоритм работает с попарными предпочтениями, а значит для его использования не требуется выявления функции полезности. В работе представлено эмпирическое исследование реализации, основанной на формальном определении алгоритма. Количественные показатели результатов экспериментов, проведенных над алгоритмом, сравниваются с другими методами машинного обучения. Так же в работе представлена модификация, расширяющая область применения алгоритма. Проведенные эксперименты показали, что алгоритм адекватно работает на реальных наборах данных.

\textbf{Ключевые слова}: обучение предпочтениям, машинное обучение, предпочтения при прочих равных, обучение ранжированию, попарные предпочтения 

\vspace{2em}

Работа состоит из 33 страниц, 3 глав, содержит 2 таблицы, 1 иллюстрацию и 32 источника. 

\newpage


Ranking problem is one of the most extensively studied field of preference learning. In this paper an implementation of the ceteris paribus preference elicitation algorithm is presented. This algorithm works with binary relations and supports purely relative data that has no utility function. Based on the formal description of the algorithm, we conduct an empirical study of its implementation. The quantitative measurement of the implementation is compared with various machine learning techniques. In addition, we propose a modification of the algorithm, which expand its scope. Results of conducted experiments proves the ability of the algorithm to work with real-world datasets.

\textbf{Keywords}: preference learning, machine learning, ceteris paribus preferences, learning to rank, pairwise preferences
	%%\chapter*{Определения, сокращения и обозначения}
\addcontentsline{toc}{chapter}{Определения, сокращения и обозначения}
	%\chapter*{Введение}
\addcontentsline{toc}{chapter}{Введение}

Обучение предпочтениям (preference learning) – одна из наиболее развиваемых областей машинного обучения. Главная задача алгоритмов выявления предпочтений – конструирование модели, основанной на обучающем наборе данных, и предсказание отношений предпочтений в случае добавления новых объектов в модель.

В настоящее время исследователи наибольшее внимание уделяют задачи ранжирования. У обучения ранжированию существует большое количество приложений. Например, розничные сети могут предсказывать предпочтения клиента, основываясь на его демографических данных, таких как пол, возраст, семейное положение и т.п. Другим примером является биоинформатика, где для упорядочивания генов применяется ранжирование, основанное на филогенетических данных. \cite{Balasubramaniyan:2005} Поисковые движки используют алгоритмы ранжирования для упорядочивания поисковой выдачи. Более того, обучение ранжированию может быть использовано для упорядочивания самих алгоритмов ранжирования\cite{Brazdil:2003}.

Основная часть данной работы посвящена различным алгоритмам выявления предпочтений. Задача таких алгоритмов заключается в ранжировании множества объектов. Обычно алгоритмы выявления предпочтений являются одним из ключевых модулей рекомендательных систем. Такие системы пытаются \enquote{угадать} предпочтение пользователя, используя информацию о его предыдущих действиях и действиях пользователей, похожих на него. Примерами таких систем являются сервисы imhonet.ru\footnote{http://imhonet.ru/about/ – о проекте Имхонет}, kinopoisk.ru\footnote{http://www.kinopoisk.ru/docs/join/ - о возможностях Кинопоиска}, last.fm\footnote{http://www.lastfm.ru/about – о проекте last.fm} и многие другие.

Основная \emph{цель} данной работы – исследование алгоритма выявления предпочтений \enquote{при прочих равных} (далее - Алгоритм), который был описан в \cite{Obiedkov:2013}. Основные \emph{задачи}:
\begin{enumerate}[itemsep=-1mm]
	\item Реализация Алгоритма на одном из прикладных языков программирования
	\item Сравнение результатов работы Алгоритма с другими методами машинного обучения
	\item Модификация Алгоритма, заключающаяся в расширении его области действия
\end{enumerate}

%\section*{Обоснование выбора алгоритма}
%TODO обоснование почему попарные предпочтения лучше семантического дифференциала

\section*{Структура работы}
Остальная часть работы организована следующим образом:
\begin{itemize}
	\item В Главе~\ref{chapter:literature} представлен обзор литературы, посвященной методам выявления предпочтений, описан теоретический базис задач ранжирования, а так же описаны алгоритмы, используемые в исследовании;
	\item Глава~\ref{chapter:theory} представляет формальное описание Алгоритма, показан пример его работы и предложена модификация;
	\item Глава~\ref{chapter:experiments} посвящена эмпирическому исследованию работы Алгоритма. В ней описан метод сравнения алгоритмов, представлены наборы данных, на которых тестировались алгоритмы, и выделены полученные результаты;
	\item Заключение подводит итоги проведенной работы.
\end{itemize}

	%\chapter{Обзор литературы}
\label{chapter:literature}

На тему задач ранжирования написано множество литературы. Помимо отдельных статьей опубликовано несколько сборников, посвященных данной проблеме. Примером такого сборника является \textit{Preference Learning} \cite{plbook:2010} Фюрнкранца и Хюллермайера, в котором авторы собрали широкий спектр различных работ на тему обучения предпочтений. Лиу в \textit{Learning to Rank for Information Retrieval} \cite{Liu:2011} представил обширные исследования задач ранжирования. Однако, основное внимание Лиу уделяет задачам извлечение информации, в то время как данная работа посвящена упорядочиванию объектов.

Фюрнкранц и Хюллермайер в \cite{plbook:Introduction:2010} представляют подробную классификацию задач ранжирования и методов для их решения. В другой работе \cite{Furnkranz:2003} авторы описывают алгоритм ранжирования, который выводит функцию полезности на основе отношений предпочтения.

Было предложено несколько методов для создания модели предпочтений. Хэддави и др. в \cite{Haddawy:2003} описали метод выявления предпочтений, основанный на простой нейросети. 
	%\chapter{Теоретические основы}
\label{chapter:theory}

%TODO: указать какие задачи еще существуют
Хотя существует множество задач в сфере выявления предпочтений, в данной работе мы рассматриваем проблему ранжирования. Данная глава посвящена теоретическим основам задачи упорядочивания. Тем не менее, от читателя ожидается знание основ теории множеств. Большая часть материала данной главы основана на \enquote{Введении} из сборника \textit{Preference Learning} Фюрнкранца и Хюллермайера\cite{plbook:Introduction:2010}.


\section{Теоретические основы задач ранжирования}

	Алгоритмы ранжирования могут отличаться друг от друга по ряду признаков. Так, часть алгоритмов работают с линейно упорядоченными множествами (полным порядком), в то время как другие работают с частичным порядком. Говорят, что отношение предпочтения является полным порядком, если про любые две альтернативы соединены отношением предпочтения. Это значит, что из каждой пары объектов\footnote{В данной работе слово \enquote{объект} используется как термин анализа формальных понятий} можно однозначно выбрать более предпочитаемый вариант (или, по крайней мере, выбрать вариант не хуже). С другой стороны, в случае частичного порядка могут существовать пары объектов, которые не связаны между собой отношением предпочтения. В таком случае мы имеем дело с частичном порядком. 
	
	Так же алгоритмы могут быть разделены по признаку возможности работы с нестрогими отношениями. Так, строгое отношение между объектами $a$ и $b$ может быть интерпретировано как \enquote{$a$ лучше $b$}, в то время как нестрогие отношения означают утверждения вида \enquote{$a$ не хуже $b$}.\cite[p.~384]{Barten:1982} Алгоритм выявления предпочтений \enquote{при прочих равных} может работать только с частичным порядком и нестрогим отношением предпочтения.

\subsection{Типы задач ранжирования}
	Фюрнкранц и Хюллермайер в \cite{plbook:Introduction:2010} выделяют три типа задач ранжирования: ''ранжирование меток" (\emph{label ranking}), ''ранжирование экземпляров'' (\emph{instance ranking}) и ''ранжирование объектов'' (\emph{object ranking}).
	
	\subsubsection{Ранжирование меток}
		Данный тип ранжирования используется в ситуациях, когда надо найти порядок фиксированного набора элементов в зависимости от предоставленного контекста. Примером такого ранжирования может является упорядочение списка товаров в зависимости от личных данных клиента.
	
	%TODO: переформулировать
	\subsubsection{Ранжирование экземпляров}
		Данный тип используется в случаях, когда необходимо каждый из объектов отнести к одному из классов из фиксированного набора. Такое ранжирование обычно используется для оценки альтернатив, таких как оценка фильма в рекомендательной системе.
	
	\subsubsection{Объектное ранжирование}
		Алгоритмы для объектного ранжирования работают с 
		относительной %TODO: использовать другое слово
		информацией, в которой отсутствуют предопределенные категории. Для данной работы это наиболее важный тип ранжирования, так как рассматриваемый нами Алгоритм работает именно с попарными отношениями объектов (относительный порядок). Пример данного типа ранжирования приведен в главе \ref{???}.
		
		Так как в данной работе наиболее часто применение находит именно объектное ранжирование, на рис. \ref{fig:object_ranking} дано его формальное определение. Алгоритм выявления предпочтений \enquote{при прочих равных} решает более узкую задачу: на вход приходит множество уже проранжированных объектов $\Z \setminus \{e\}$ и объект $e$. Работа алгоритма заключается в упорядочении множества $\Z$.
		
		\begin{figure}[h]
			\hrule
			\begin{description}[nosep]
				\item[Дано:] \null\leavevmode
				\begin{itemize}[itemsep=0pt,leftmargin=2ex,label=\textbf{---}]
					\item множество (возможно, бесконечное) объектов $\Z$ (обычно (но не обязательно) каждый объект представлен в виде вектора атрибутов)
					\item конечное множество попарных предпочтений $x_i \succ x_j, (x_i, x_j) \in \Z \times \Z$
				\end{itemize}
				\item[Найти:] \null\leavevmode
				\begin{itemize}[itemsep=0pt,leftmargin=2ex,label=\textbf{---}]
					\item функцию ранжирования $f(\cdot)$, которая принимает на вход множество объектов и возвращает перестановку (ранжирует) этого множества
				\end{itemize}
			\end{description} 
			\hrule
			\caption{\it Определение объектного ранжирования (определение из \cite[Рис.~3]{plbook:Introduction:2010})}
			\label{fig:object_ranking}
		\end{figure}
	
	\subsection{Подходы к реализации}
		В \cite{plbook:Introduction:2010} представлено четыре подхода к реализации алгоритмов ранжирования. 
		Первый вариант – найти \emph{функцию полезности} и построить полный порядок основываясь на значениях, полученных с помощью этой функции. Такой подход применим ко всем трем типам задач ранжирования. 
		Второй подход заключается в \emph{выявлении зависимостей} в попарных отношениях предпочтения. Хотя часто подобный способ и создает более простую модель, в некоторых случаях может возникать нарушение транзитивности\cite[стр.~10]{plbook:Introduction:2010}. 
		Третий подход называется \emph{обучение предпочтениям, основанное на модели} и используется в случаях, когда известны некоторые ограничения об отношениях предпочтения. 
		Последний подход, который выделяют авторы, называется \emph{локальное агрегирование предпочтений}. Основная идея данного метода заключается в поиске ближайших \enquote{соседей} для данного объекта.
		
		Очевидно, что описанные подходы не являются взаимоисключающими. Алгоритм выявления предпочтений \enquote{при прочих равных} является примером смешения подходов. С одной стороны, Алгоритм имеет дело с бинарными отношениями и подходит к типу ''выявления зависимостей''. С другой стороны, он работает с отношениями \enquote{при прочих равных}, что накладывает определенные ограничения на результирующую модель. Следовательно, Алгоритм так же может быть отнесен к обучению, ''основанному на модели''.
	
	
\section{Алгоритм выявления предпочтений \enquote{при прочих равных}}

	Данный раздел основан на работе Объедкова \cite{Obiedkov:2013}, в которой представлен Алгоритм. Данный раздел состоит из двух секций: в первом представлен пример входных данных и описан ожидаемый результат работы Алгоритма, во втором представлено формальное определение Алгоритма.
	
	\subsection{Пример}
		На рис. \ref{fig:pcxt} представлен пример контекста предпочтений. Объекты $c_1, \dots, c_5$ представляют машины, у каждой из которых есть свой набор признаков. Например, $c_2$ является белым внедорожником с темным интерьером. Диаграмма на правой стороне показывает отношение предпочтений какого-то пользователя: $c_1$ не хуже, чем $c_2$ и $c_3$; $c_5$ не хуже, чем $c_1$; $c_2$ и $c_3$ несравнимы; и так далее. 
		\begin{figure}
			\begin{center} 
				\cars \prefs
				\caption{\it Пример контекста предпочтений \cite[Рис.~1.1]{Obiedkov:2013}}
				\label{fig:pcxt}	
			\end{center} 
		\end{figure} 
		
		Задача алгоритма предсказать отношения предпочтения при добавлении нового объекта в контекст. В рассматриваемом примере в контекст может быть добавлен красный миниван с ярким интерьером, и Алгоритм должен выявить отношения предпочтения для добавленной машины, попарно сравнив её с $c_1, \dots, c_5$. 
	
	\subsection{Определения}
		Представленный в данной работе алгоритм оперирует сущностями анализа формальных понятий\cite{Ganter:1999}, а так же понятиями, определенными в \cite{Obiedkov:2012:preferences,Obiedkov:2012:modeling}. В данном разделе представлены эти определения.
		
		
		\begin{definition}
			\emph{(Формальный) контекст} – это тройка $\context = (G, M, I)$, где $G$ называется множеством \emph{объектов}, $M$ называется множеством \emph{признаков}, и бинарное отношение ${I \subseteq G \times M}$ указывает на принадлежность признаков к каждому из объектов.
		\end{definition}
		
		Формаьлный контекст может быть визуализирован с помощью таблицы, такой как Рис. \ref{fig:pcxt}.
		
		Для множеств $A \subseteq G$ и $B \subseteq M$ следующим образом определены \emph{операторы Галуа} (derivation operators) $(\cdot)'$:
		\begin{center}
			$A'=\{m \in M \mid \forall g \in A (g I m)\}$
			
			$B'=\{g \in G \mid \forall m \in B (g I m)\}$
		\end{center}
		$A'$ -– это набор признаком, которые есть у всех объектов множества $A$, а $B'$  –- набор объектов, каждый из которых содержит все признаки множества $B$. Пусть $g \in G$ и $m \in M$, тогда множества $\{g\}'$ и $\{m\}'$ называются \emph{содержание объекта} (object intent) и \emph{объем признака} (attribute extent), соответственно. Иногда их обозначают как $g'$ и $m'$.
		
		\begin{definition}
			\emph{Контекст предпочтений} $\PP = (G, M, I, \leq)$ это формальный контекст $(G, M, I)$ с рефлексивным и транзитивным отношением предпочтения $\leq$, определенным над $G$ (то есть, $\leq$ – предпорядок). Мы пишем $g < h$, если $g \leq h$ и $h \not\leq g$.
		\end{definition}
		
		Пример такого контекста представлен на Рис. \ref{fig:pcxt}.
		
		\begin{definition}
			Множество признаков $B \subseteq M$ \emph{предпочитается \enquote{при прочих равных}} множеству признаков $A \subseteq M$ по отношению ко множеству признаков $C \subseteq M$ в контексте предпочтений $\PP = (G, M, I, \leq)$ если 
			\[\forall g \in A' \forall h \in B'(\{g\}' \cap C = \{h\}' \cap C \to g \leq h).\]
			В таком случае мы говорим что предпочтение \emph{при прочих равных} $A \cp{C} B$ является \emph{валидным} или \emph{имеет место} в $\PP$ и обозначаем это как $\PP \models \ABC$.
		\end{definition}
	
	\subsection{Описание алгоритма}
	%TODO: дописать
		Алгоритм \ref{algo:prediction} является реализацией алгоритма выявления предпочтений \enquote{при прочих равных}, написанной на псевдокоде. На шестой строке алгоритм вызывает подпрограмму, задача который убедиться, что никакие пары контекста не противоречат данному отношению.
		
		\begin{algorithm}
			\caption{\algname{Предсказание предпочтения}$(A, B, \PP)$ \cite[Алг.~1]{Obiedkov:2013}}
			\label{algo:prediction}
			\begin{algorithmic}[1]
				\REQUIRE Object intents $A, B \subseteq M$ and a preference context $\PP = (G, M, I, \leq)$.
				\ENSURE \TRUE, if $\PP$ supports $\DEF$ for some $D \subseteq A, E \subseteq B,$ and $F \subseteq M$ such that $F \cap A = F \cap B$; \FALSE, otherwise.
				\item[]
				\FORALL{$g \in G$}
				\STATE $D := A \cap g'$
				\FORALL{$h \in G \setminus \{g\}$ such that $g \leq h$}
				\STATE $E := B \cap h'$
				\STATE $F := (M \setminus (A \vartriangle B)) \cap (M \setminus (g' \vartriangle h'))$
				\IF{$\PP \models \DEF$}
				\RETURN \TRUE
				\ENDIF
				\ENDFOR
				\ENDFOR
				\RETURN \FALSE
			\end{algorithmic}
		\end{algorithm}
		\begin{algorithm}
			\caption{\algname{Check Preference}$(\DEF, \PP)$ \cite[Alg.~2]{Obiedkov:2013}}
			\label{algo:check}
			\begin{algorithmic}
				\REQUIRE A ceteris paribus preference $\DEF$ over $M$ and a preference context $\PP = (G, M, I, \leq)$.%; assume that attribute extents, $m'$ for $m \in M$, are precomputed.
				\ENSURE \TRUE, if $\PP \models \DEF$; \FALSE, otherwise.
				\STATE
				\STATE $X := \bigcap_{m \in D}m'$
				\STATE $Y := \bigcap_{m \in E}m'$
				\FORALL{$g \in X$}
				\FORALL{$h \in Y$}
				\IF {$g \not\leq h$ and $g' \cap {F} = h' \cap {F}$}
				\RETURN \FALSE
				\ENDIF
				\ENDFOR
				\ENDFOR
				\RETURN \TRUE
			\end{algorithmic}
		\end{algorithm}
		
	%\chapter{Эксперименты}
\label{chapter:experiments}
В данной главе представлена экспериментальная часть работы. Сравнение результатов работы алгоритма выявления предпочтений \enquote{при прочих равных} с другими методами обучения предпочтениям является ключевой частью данного исследования.

Глава состоит из четырех разделов. В первом описаны наборы данных, на которых сравнивались алгоритмы. Во втором разделе описаны сами алгоритмы, которые были рассмотрены в данной работе. В третьем описана методика сравнения алгоритмов. И в заключительной части представлены результаты экспериментов.

\section{Входные данные}
	В данной работе представлена
	апробация Алгоритма проводилась на трех наборах данных: тестовые данные с предпочтениями об автомобилях, реальный набор данных о пользовательских предпочтениях в автомобилях %TODO: переформулировать
	и реальный набор данных о предпочтениях в суши. Ниже подробно описан каждый из наборов данных.
	
	\subsection{Искусственный набор данных}
		Искусственный набор взят из \cite{Obiedkov:2013}. Этот набор данных состоит из семи объектов, пять из которых представлены на рис.~\ref{fig:pcxt}. Еще два объекта:
		$c_6={minivan, red, bright}$ и $c_7={SUV, red, bright}$. Эти данные не много могут сказать о качестве работы алгоритма, но их можно использовать для базовой проверки его возможностей.
	
	\subsection{Реальный набор данных: автомобили}
		Первый реальный набор данных собран Аббаснеджадом и др. для \cite{dataset:Abbasnejad:2013}. Для сбора пользовательских предпочтений авторы использовали краудсорсинговую платформу Amazon Mechanical Turk\footnote{mturk.com – информация о проекте Amazon Mechanical Turk}. Набор данных состоит из двух экспериментов\footnote{Данные размещены по адресу: http://users.cecs.anu.edu.au/~u4940058/CarPreferences.html}: в первом была собрана информация от 60 пользователей о 10 машинах, во втором – от 60 пользователей о 20 машинах. Причем во втором эксперименте информации о каждой из машин было больше. Далее подробно описана информация, представленная в наборе данных.
		
		\vspace{1em}
		
		\noindent Характеристики пользователей:
		\vspace{-0.7em}
		\begin{itemize}[itemsep=-1.5mm]
			\item ID: уникальный идентификатор пользователя
			\item Образование: отсутствие ответа (0), старшая школа (1), бакалавриат (2), PhD (3)
			\item Возраст: отсутствие ответа (0), меньше 25 (1), от 25 до 30 (2), от 30 до 35 (3), больше 40 (4)
			\item Пол: отсутствие ответа (0), мужской (1), женский (2)
			\item Регион: отсутствие ответа (0), юг (1), запад (2), северо-восток (3), средний запад (4)
			\item Количество правильных ответов на контрольные вопросы: 3, 4 или 5
		\end{itemize}
		Данные собирались среди американской аудитории. Каждому пользователю было задано по пять контрольных вопросов, каждый из которых является одним из пользовательских предпочтений в обратном порядке (например, если пользователь указал, что $o_1 < o_5$, то контрольный вопрос имеет вид $o_5 < o_1?$, где $o_1$ и $o_5$ – какие-то объекты). На основе количества правильных ответов на контрольные вопросы можно судить о согласованности пользовательских предпочтений.
		
		\vspace{1em}
		
		\noindent Признаки машин:
		\vspace{-0.7em}
		\begin{itemize}[itemsep=-1.5mm]
			\item Тип кузова: седан (1), SUV\footnote{SUV (англ. Sport Utility Vehicle или Suburban Utility Vehicle) – подобие внедорожника} (2), хэтчбек (3)
			\item Коробка передач: ручная (1), автоматическая (2)
			\item Объем двигателя: 2.5L, 3.5L, 4.5L, 5.5L, 6.2L
			\item Тип топлива: гибрид (1), не гибрид (2)
			\item Количество ведущих колес: все колеса ведущие (AWD, 4x4) (1), передние колеса ведущие (FWD) (2)
		\end{itemize} 
		В первом эксперименте не были представлены хэтчбеки, а так же не использовалась информация о количестве ведущих колес.
	
	\subsection{Реальный набор данных: суши}
	
\section{Рассматриваемые алгоритмы}

\section{Методика}

$k$-fold cross-validation

\section{Результаты}
	%\chapter*{Заключение}
\addcontentsline{toc}{chapter}{Заключение}
\label{ch:ending}

Основной задачей данной работы было экспериментальное исследование алгоритма выявления предпочтений \enquote{при прочих равных}, которое заключается в апробации Алгоритма на реальных наборах данных, а так же в сравнении результатов экспериментов с результатами уже существующих методов. В ходе работы на языке Java был реализован Алгоритм и проведено исследование качества его работы на основе двух реальных наборов данных. Для оценки качества работы Алгоритма были проведены эксперименты над теми же наборами данных, но с использованием других алгоритмов: C4.5, наивный Байесовский классификатор и клссификатор, основанный на Байесовской сети.

В ходе работы разработаны формальные описания нескольких модификаций Алгоритма, а также проведено эмпирическое исследование работы их реализаций. Часть модификаций являются простыми адаптациями, задачей которых было не меняя семантики оригинального Алгоритма повысить качественные показатели его работы при проверке на реальных данных. Модификация, основанная на предикатах, изменяет семантику Алгоритма, расширяя его возможности. 

Рассматривается более общий класс предпочтений, где мы используем не только отношения равенства, но и прочие отношения между признаками.

Проведенные эксперименты показали, что алгоритм выявления предпочтений \enquote{при прочих равных} может быть успешно использован на реальных данных. В первой серии экспериментов показатели результатов работы простых модификаций оказались сопоставимы с широко используемыми методами машинного обучения, а модификация, основанная на предикатах, по показателям полноты превзошла C4.5 с числовыми признаками (\emph{C4.5 paired, num}), который показывал лучшие результаты в исследовании. Вторая серия экспериментов, которые проводились на наборе данных с суши, показала превосходство Алгоритма над рассматриваемыми методами машинного обучения. 

	%\bibliography{bibliography} 
	%\addcontentsline{toc}{chapter}{Список использованных источников}
	%\appendix
\chapter*{Приложение А. Исходный код программы}
\addcontentsline{toc}{chapter}{Приложение А. Исходный код программы}

Исходный код реализации алгоритма выявления предпочтений \enquote{при прочих равных} и его модификаций, а также код для проведения описанных экспериментов, представлен на прилагаемом к данной работе компакт-диске.

\appendix
\chapter*{Приложение Б. Дополнительные эксперименты}
\addcontentsline{toc}{chapter}{Приложение Б. Дополнительные эксперименты}

В таблице \ref{tbl:extra_experiments} представлены результаты экспериментов над набором данных о предпочтениях в суши. Серым фоном выделены результаты дополнительных экспериментов.

Как видно из таблицы, на наборе данных о суши байесовская сеть и наивный байесовский классификатор превосходят остальные методы выявления предпочтений по большинству показателей. По показателям правильности и полноты байесовские классификаторы оказываются лучше других алгоритмов, однако по точности реализация Алгоритма оказывается лучше.

\begin{sidewaystable}[ph!]
	\centering
	\tablecaption{Суши: результаты дополнительных экспериментов}
	\begin{tabular}{|l|ccc|ccc|}
		\hline
		\multirow{2}{*}{Метод}   & \multicolumn{3}{c|}{1-out}                             & \multicolumn{3}{c|}{2-out}         \\ \cline{2-7}  
		& правильность \%  & точность \%      & полнота \%       & правильность \%   & точность \%       & полнота \% \rule{0pt}{2.4ex} \\ \hline
		CP, mod    				 & $63.93 \pm 6.01$ & $77.94 \pm 11.76$  & $43.44 \pm 5.87$ & $55.57 \pm 7.75$  & $66.33 \pm 21.63$  & $26.28 \pm 8.83$ \rule{0pt}{2.4ex} \\ 
		CPs, mod				 & $70.44 \pm 8.97$  & $71.31 \pm 9.06$  & $93.12 \pm 4.49$ & $60.08 \pm 13.6$  & $60.42 \pm 14.23$  & $90.63 \pm 8.02$ \\
		CPfs, mod				 & $69.7 \pm 8.9$ & $70.76 \pm 9.03$ & $91.44 \pm 5.91$  & $59.42 \pm 12.87$   & $59.95 \pm 13.95$  & $85.05 \pm 10.59$ \\
		C4.5 paired, num 	     & $63.95 \pm 12.82$ & $67.25 \pm 16.7$ & $64 \pm 21.74$ & $58.39 \pm 13.43$  & $60.07 \pm 19.2$ & $54.44 \pm 21.87$ \\
		\rowcolor{lightgray} 
		Naive Bayes  			 & $74.39 \pm 4.47$  & $74.5 \pm 4.48$ & $99.42 \pm 0.96$  & $60.31 \pm 14.82$  & $60.71 \pm 15.5$   & $92.43 \pm 4.95$ \\ 
		\rowcolor{lightgray} 
		Bayes Net  				 & $74.42 \pm 4.57$  & $74.56 \pm 4.58$ & $99.26 \pm 1.09$ & $60.27 \pm  14.76$ & $60.8 \pm 15.61$ & $90.48 \pm 4.78$ \\ 
		\hline
	\end{tabular}
	\label{tbl:extra_experiments}
\end{sidewaystable}
	%\bibliographystyle{ieeetr}
\end{document}