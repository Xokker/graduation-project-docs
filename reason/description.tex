\chapter*{Введение}
\addcontentsline{toc}{chapter}{Введение}

Обучение предпочтениям (preference learning) – одна из наиболее развиваемых областей машинного обучения. Главная задача алгоритмов выявления предпочтений – конструирование модели, основанной на тренировочном наборе данных, и предсказание отношений предпочтений в случае добавления новых объектов в модель.

В настоящее время исследователи наибольшее внимание уделяют задачи ранжирования. У обучения ранжированию существует большое количество приложений. Например, розничные сети могут предсказывать предпочтения клиента, основываясь на его демографических данных, таких как пол, возраст, семейное положение и т.п. Другим примером является биоинформатика, где для упорядочивания генов применяется ранжирование, основанное на филогенетических данных. \cite{Balasubramaniyan:2005} Поисковые движки используют алгоритмы ранжирования для упорядочивания поисковой выдачи. Более того, обучение ранжированию может быть использовано для упорядочивания самих алгоритмов ранжирования\cite{Brazdil:2003}.

Основная часть работы - изучение работы алгоритмов выявления предпочтений. Задача таких алгоритмов заключается в ранжировании множества объектов. Обычно алгоритмы выявления предпочтений являются одним из ключевых модулей рекомендательных систем. Такие системы пытаются \enquote{угадать} предпочтение пользователя, используя информацию о его предыдущих действиях и действиях пользователей, похожих на него. Примерами таких систем являются сервисы imhonet.ru\footnote{http://imhonet.ru/about/ – о проекте Имхонет}, kinopoisk.ru\footnote{http://www.kinopoisk.ru/docs/join/ - о возможностях Кинопоиска}, last.fm\footnote{http://www.lastfm.ru/about – о проекте last.fm} и многие другие.

Одним из подходов к выявлению предпочтений является алгоритм, предложенный Объедковым~С.А. в \cite{Obiedkov:2013}. В исследовательской работе необходимо реализовать алгоритм и исследовать его работу на реальных данных. Так же необходимо найти пути модификации алгоритма и реализовать их.
