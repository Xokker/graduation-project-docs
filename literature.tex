\chapter{Обзор литературы}
\label{chapter:literature}

На тему задач ранжирования написано множество литературы. Помимо отдельных статьей опубликовано несколько сборников, посвященных данной проблеме. Примером такого сборника является \textit{Preference Learning} \cite{plbook:2010} Фюрнкранца и Хюллермайера, в котором авторы собрали широкий спектр различных работ на тему обучения предпочтений. Лиу в \textit{Learning to Rank for Information Retrieval} \cite{Liu:2011} представил обширные исследования задач ранжирования. Однако, основное внимание Лиу уделяет задачам извлечение информации, в то время как данная работа посвящена упорядочиванию объектов.

Фюрнкранц и Хюллермайер в \cite{plbook:Introduction:2010} представляют подробную классификацию задач ранжирования и методов для их решения. Подробнее данная классификация описана в Главе \ref{chapter:theory}. В другой работе \cite{Furnkranz:2003} авторы описывают алгоритм ранжирования, который выводит функцию полезности на основе отношений предпочтения.

Было предложено несколько методов для создания модели предпочтений. Хэддави и др. в \cite{Haddawy:2003} описали метод выявления предпочтений, основанный на простой нейросети. Жадный алгоритм, находящий аппроксимацию задачи ранжирования представил Кохен в \cite{Cohen:1999}.

Целый ряд алгоритмов машинного обучения был адаптирован для задач ранжирования. Примером такой адаптации является исследование, проведенное Женгом и др. в \cite{Zheng:2007}, где была предложен метод сведения задачи ранжирования к задачи регрессии. В дополнение, Бургерс и др. в \cite{Burges:2005} описали подход к задачи ранжирования, основанный на алгоритме градиентного спуска.