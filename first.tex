\chapter{Теоретические основы задач ранжирования}
\label{chapter:first}

%TODO: указать какие задачи еще существуют
Хотя существует множество задач в сфере выявления предпочтений, в данной работе мы рассматриваем проблему ранжирования. Данная глава посвящена теоретическим основам задачи упорядочивания. Тем не менее, от читателя ожидается знание основ теории множеств и анализа формальных понятий (FCA, Formal Concept Analysis). Большая часть материала данной главы основана на \enquote{Введении} из сборника \textit{Preference Learning} Фюрнкранца и Хюллермайера\cite{plbook:Introduction:2010}.

Алгоритмы ранжирования могут отличаться друг от друга по ряду признаков. Так, часть алгоритмов работают с линейно упорядоченными множествами (полным порядком), в то время как другие работают с частичным порядком. Говорят, что отношение предпочтения является полным порядком, если про любые две альтернативы соединены отношением предпочтения. Это значит, что из каждой пары объектов\footnote{В данной работе слово \enquote{объект} используется как термин анализа формальных понятий} можно однозначно выбрать более предпочитаемый вариант (или, по крайней мере, выбрать вариант не хуже). С другой стороны, в случае частичного порядка могут существовать пары объектов, которые не связаны между собой отношением предпочтения. В таком случае мы имеем дело с частичном порядком. 

Так же алгоритмы могут быть разделены по признаку возможности работы с нестрогими отношениями. Так, строгое отношение между объектами $a$ и $b$ может быть интерпретировано как \enquote{$a$ лучше $b$}, в то время как нестрогие отношения означают утверждения вида \enquote{$a$ не хуже $b$}.\cite[p.~384]{Barten:1982} Алгоритм выявления предпочтений \enquote{при прочих равных} может работать только с частичным порядком и нестрогим отношением предпочтения.

Данная глава состоит из 3 разделов. В первом дается литературный обзор. Во втором рассматриваются существующие подходы к выявлению предпочтений. В третьем описывается подход, который используется в данной работе.
