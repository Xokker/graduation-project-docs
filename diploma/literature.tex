\chapter{Обзор существующих решений}
\label{chapter:literature}

На тему задач ранжирования написано множество литературы. Помимо отдельных статьей опубликовано несколько сборников, посвященных данной проблеме. Примером такого сборника является \textit{Preference Learning} \cite{plbook:2010} Фюрнкранца и Хюллермайера (F{\"{u}}rnkranz \& H{\"{u}}llermeier), в котором авторы собрали широкий спектр различных работ на тему обучения предпочтений. Лиу (Liu) в \textit{Learning to Rank for Information Retrieval} \cite{Liu:2011} представил обширные исследования задач ранжирования. Однако, основное внимание Лиу уделяет задачам извлечение информации, в то время как данная работа посвящена упорядочиванию объектов.

Фюрнкранц и Хюллермайер в \cite{plbook:Introduction:2010} представляют подробную классификацию задач ранжирования и методов для их решения. Подробнее данная классификация описана в Главе \ref{chapter:theory}. В другой работе \cite{Furnkranz:2003} авторы описывают алгоритм ранжирования, который выводит функцию полезности на основе отношений предпочтения.

Было предложено несколько методов для создания модели предпочтений. Хэддави (Haddawy et al.) \cite{Haddawy:2003} описал метод выявления предпочтений, основанный на простой нейросети. Жадный алгоритм, находящий аппроксимацию задачи ранжирования представил Кохен (Cohen) в \cite{Cohen:1999}.

Целый ряд алгоритмов машинного обучения был адаптирован для задач ранжирования. Примером такой адаптации является исследование, проведенное Женгом (Zheng et al.) \cite{Zheng:2007}, где был предложен метод сведения задачи ранжирования к задачи регрессии. В дополнение, Бургерс (Burges et al.) \cite{Burges:2005} описал подход к задачи ранжирования, основанный на алгоритме градиентного спуска.

Камишима (Kamishima) \cite{Kamishima:2003} описывает проведенное им исследование, которое доказывает несостоятельность применения семантического дифференциала \cite{Osgood:1957} к пользовательским предпочтениям. Данные, собранные автором, показывают, что оценка альтернатив с помощью числовой шкалы не совпадает с более интуитивным методом сбора предпочтений – ранжированием. Отметим, что ранжированный список предпочтений и попарные предпочтения, рассматриваемые в настоящей работе, могут быть тривиальным образом трансформированы друг в друга.

%TODO: указать какие задачи еще существуют
Хотя существует множество задач в сфере выявления предпочтений, в данной работе мы рассматриваем проблему ранжирования. Данная глава посвящена теоретическим основам задачи упорядочивания. Большая часть материала данной главы основана на \enquote{Введении} из сборника \textit{Preference Learning} Фюрнкранца и Хюллермайера\cite{plbook:Introduction:2010}.

Остальная часть данной главы организована следующим образом: в первом разделе описан теоретический базис задач ранжирования, во втором представлены уже существующие алгоритмы, которые участвуют в экспериментах в Главе~\ref{chapter:experiments}.

\section{Теоретические основы задач ранжирования}

	Алгоритмы ранжирования могут отличаться друг от друга по ряду признаков. Так, часть алгоритмов работают с линейно упорядоченными множествами (полным порядком), в то время как другие работают с частичным порядком. Говорят, что отношение предпочтения является полным порядком, если любые две альтернативы связаны отношением предпочтения. Это значит, что из каждой пары объектов\footnote{В данной работе слово \enquote{объект} используется как термин анализа формальных понятий. Подробнее см. раздел \ref{subsection:definitions}} можно однозначно выбрать более предпочитаемый вариант (или, по крайней мере, выбрать вариант не хуже). С другой стороны, частичный порядок допускает существование пар объектов, которые не связаны между собой отношением предпочтения. 
	\begin{definition}
		\label{def:total_order}
		Отношение порядка $\leq$, для которого выполняется условие
		$\forall x,y\in X \quad  x\leq y \vee y \leq x$,
		называется \emph{полным порядком}.
	\end{definition}
	
	Предпочтения могут находиться в различных отношениях: предпорядка (preorder), нестрого порядка (nonstrict order) или строгого порядка (strict order). Так, строгое отношение между объектами $a$ и $b$ может быть интерпретировано как \enquote{$a$ лучше $b$}, в то время как нестрогие отношения означают утверждения вида \enquote{$a$ не хуже $b$}\cite[с.~384]{Barten:1982}. Предпорядок – это нестрогий порядок без свойства антисимметричности. Определения \ref{def:preorder}, \ref{def:nonstrict_order} и \ref{def:strict_order} представляют формальные описания данных понятий.
	
	\begin{definition}
		\label{def:preorder}
		Отношение $\leq$ называется \emph{предпорядком} (или квазипорядком) на множестве $S$, если оно удовлетворяет следующим условиям\cite{harel:2000}:
		\begin{itemize}[itemsep=-1.5mm]
			\item Рефлексивность: $a \leq a \quad \forall a \in S$
			\item Транзитивность: $a \leq b\: \wedge\: b \leq c \implies a \leq c$ 
		\end{itemize}
	\end{definition}	
	\begin{definition}
		\label{def:nonstrict_order}
		Множество $S$ находится в \emph{нестрогом порядке}, если это множество – предпорядок, и на нем выполняется свойство антисимметричности\cite{Skiena:1991}: $a \leq b\; \wedge\; b \leq a \implies a = b$.
	\end{definition}
	\begin{definition}
		\label{def:strict_order}
		Множество $S$ находится в \emph{строгом порядке}, если на этом множестве выполняются следующие свойства:
		\begin{itemize}[itemsep=-1.5mm]
			\item Антирефлексивность: $ a \nless a \quad \forall a \in S$
			\item Асимметричность: $a < b \implies b \nless a \quad \forall \, (a, b) \in S$
			\item Транзитивность: $a < b\: \wedge\: b < c \implies a < c$
		\end{itemize}
		Если в нестрогом порядке условие рефлексивности заменить на антирефлексивность, получится строгий порядок.
	\end{definition}
	
	Алгоритм выявления предпочтений \enquote{при прочих равных} работает с предпорядками.
	
	\subsection{Типы задач ранжирования}
	Фюрнкранц и Хюллермайер в \cite{plbook:Introduction:2010} выделяют три типа задач ранжирования: ``ранжирование меток" (\emph{label ranking}), ``ранжирование экземпляров'' (\emph{instance ranking}) и ``ранжирование объектов'' (\emph{object ranking}).
	
	\subsubsection{Ранжирование меток}
	Данный тип ранжирования используется в ситуациях, когда надо найти порядок фиксированного набора элементов в зависимости от предоставленного контекста. Примером такого ранжирования может является упорядочение списка товаров в зависимости от личных данных покупателя.
	
	\begin{figure}[h!]
		\hrule
		\begin{description}[nosep]
			\item[Дано:] \null\leavevmode
			\begin{itemize}[itemsep=0pt,leftmargin=2ex,label=\textbf{---}]
				\item обучающий набор данных $\{\bm{x}_\ell \, | \, \ell = 1,2,\dotsc,n\} \subseteq \mathcal{X} $ (обычно, но не обязательно, каждый экземпляр представлен в виде вектора атрибутов)
				\item множество меток $\mathcal{Y} = \{y_i\,|\,i = 1,2,\dotsc,k\}$
				\item для каждого экземпляра тренировочного набора данных $\bm{x}_\ell$: множество попарных предпочтений в виде $y_i \succ_{\bm{x}_\ell} y_j $ (данная нотация читается как ``для данного $\bm{x}_\ell$  метка $y_i$ предпочитается метке $y_j$'')
			\end{itemize}
			\item[Найти:] \null\leavevmode
			\begin{itemize}[itemsep=0pt,leftmargin=2ex,label=\textbf{---}]
				\item функцию ранжирования, которая отображает каждый $\bm{x} \in \mathcal{X}$ в перестановку $\succ_{\bm{x}_\ell}$ множества $\mathcal{Y}$
			\end{itemize}
		\end{description} 
		\hrule
		\caption*{\textit{Ранжирование меток}\cite[с.~4]{plbook:Introduction:2010}}
		\label{fig:label_ranking}
	\end{figure}
	Для данного примера $\bm{x}_\ell$ будет является вектором признаков покупателя. Например, он может содержать его возраст, образование, семейный статус и т.п. $\mathcal{Y}$ будет множеством товаров, которые необходимо ранжировать.
	
	%TODO: переформулировать
	\subsubsection{Ранжирование экземпляров}
	Данный тип используется в случаях, когда необходимо каждый из объектов отнести к одному из классов из фиксированного набора. Такое ранжирование обычно используется для оценки альтернатив, таких как оценка фильма в рекомендательной системе.
	
	\begin{figure}[h!]
		\hrule
		\begin{description}[nosep]
			\item[Дано:] \null\leavevmode
			\begin{itemize}[itemsep=0pt,leftmargin=2ex,label=\textbf{---}]
				\item набор тренировочных данных $\{\bm{x}_\ell \, | \, \ell = 1,2,\dots,n\} \subseteq \mathcal{X} $ (обычно, но не обязательно, каждый экземпляр представлен в виде вектора атрибутов)
				\item множество меток $\mathcal{Y} = \{y_i\,|\,i = 1,2,\dotsc,k\}$ и их порядок $y_1 < y_2 < \dotsb < y_k$ 
				\item соответствие каждого экземпляра тренировочного набора данных $\bm{x}_\ell$ метке $y_{i}$
			\end{itemize}
			\item[Найти:] \null\leavevmode
			\begin{itemize}[itemsep=0pt,leftmargin=2ex,label=\textbf{---}]
				\item функцию, которая позволяет ранжировать множество экземпляров $\{\bm{x}_j\}^t_{j=1}$ согласно (неизвестному) уровню предпочтений
			\end{itemize}
		\end{description} 
		\hrule
		\caption*{\textit{Ранжирование экземпляров}\cite[с.~6]{plbook:Introduction:2010}}
		\label{fig:instance_ranking}
	\end{figure}
	Для приведенного примера множество $\mathcal{Y}$ может включать все возможные оценки (например, от 1 до 10), а экземпляры $\bm{x}_1, \bm{x}_2, \dots, \bm{x}_n$ будут являться фильмами. В соответствие каждому фильму будет поставлена одна из оценок из $\mathcal{Y}$.
	
	\subsubsection{Объектное ранжирование}
	Алгоритмы для объектного ранжирования работают с 
	относительной %TODO: использовать другое слово
	информацией, в которой отсутствуют предопределенные категории. Для данной работы это наиболее важный тип ранжирования, так как  Алгоритм работает именно с попарными отношениями объектов (относительный порядок). Пример данного типа ранжирования приведен в разделе \ref{subsection:example}.
	
	Ниже представлено формальное определение объектного ранжирования. Алгоритм выявления предпочтений \enquote{при прочих равных} решает более узкую задачу: на вход приходит множество уже проранжированных объектов $\Z \setminus \{e\}$ и объект $e$. Работа алгоритма заключается в упорядочении множества $\Z$.
	
	\begin{figure}[h!]
		\hrule
		\begin{description}[nosep]
			\item[Дано:] \null\leavevmode
			\begin{itemize}[itemsep=0pt,leftmargin=2ex,label=\textbf{---}]
				\item множество (возможно, бесконечное) объектов $\Z$ (обычно, но не обязательно, каждый объект представлен в виде вектора признаков)
				\item конечное множество попарных предпочтений $\bm{x}_i \succ \bm{x}_j, \: (\bm{x}_i, \bm{x}_j) \in \Z \times \Z$
			\end{itemize}
			\item[Найти:] \null\leavevmode
			\begin{itemize}[itemsep=0pt,leftmargin=2ex,label=\textbf{---}]
				\item функцию ранжирования $f(\cdot)$, которая принимает на вход множество объектов и возвращает перестановку (ранжирует) этого множества
			\end{itemize}
		\end{description} 
		\hrule
		\caption*{\textit{Объектное ранжирование}\cite[с.~7]{plbook:Introduction:2010}}
		\label{fig:object_ranking}
	\end{figure}
	
	\subsection{Подходы к реализации}
	В \cite{plbook:Introduction:2010} представлено четыре подхода к реализации алгоритмов ранжирования. 
	Первый вариант – выявить \emph{функцию полезности} (learning utility function) и построить полный порядок, основываясь на значениях, полученных с помощью этой функции. Такой подход применим ко всем трем типам задач ранжирования. 
	Второй подход заключается в \emph{выявлении зависимостей} в попарных отношениях предпочтения (learning preference relations). Хотя часто подобный способ и создает более простую модель, в некоторых случаях может возникать нарушение транзитивности\cite[с.~10]{plbook:Introduction:2010}. 
	Третий подход называется \emph{обучение предпочтениям, основанное на модели} (model-based preference learning). Он используется в случаях, когда известны некоторые ограничения об отношениях предпочтения. 
	Последний подход, который выделяют авторы, называется \emph{локальное агрегирование предпочтений} (local aggregation of preferences). Основная идея данного метода заключается в поиске ближайших \enquote{соседей} для данного объекта.
	
	Очевидно, что описанные подходы не являются взаимоисключающими. Алгоритм выявления предпочтений \enquote{при прочих равных} является примером смешения подходов: с одной стороны, Алгоритм имеет дело с бинарными отношениями и подходит к типу ``выявления зависимостей''; с другой стороны, он работает с отношениями \enquote{при прочих равных}, что накладывает определенные ограничения на результирующую модель. Следовательно, Алгоритм так же может быть отнесен к обучению, ``основанному на модели''.

\section{Рассматриваемые алгоритмы}

	Как описано в Введении, первоочередная задача данной работы – сравнение Алгоритма \ref{algo:prediction} с другими методами машинного обучения. В сравнении участвуют методы классифицирующих деревьев, а так же Байесовские методы. В разделах \ref{subsec:c4.5} -- \ref{subsec:bayes_net} подробно описаны эти алгоритмы.
	
	\subsection{C4.5}
	\label{subsec:c4.5}
		Алгоритм C4.5 был впервые представлен Квинланом (Quinlan) в \cite{Quinlan:1993} в 1993 году, и с тех пор стал одним из самых популярных реализаций деревьев принятия решений. С4.5 является расширением алгоритма ID3 \cite{Quinlan:1986}, так же разработанным Квинланом в 1968 году. Оба этих алгоритма основаны на расчете информационной энтропии. 
		
		C4.5 предназначен для построение классифицирующих деревьев. Имея значения признаков обучающей выборки, алгоритм на каждом шаге производит ветвление по признаку, которые дает максимальный прирост информации. По сравнению с ID3, C4.5 производит ряд оптимизаций, таких как отсечение ветвей (pruning). Кроме того, в C4.5 была добавлена поддержка числовых признаков.
	
	\subsection{Наивный Байесовский классификатор}
	\label{subsec:naive_bayes}
		Наивный Байесовский классификатор – это простой статистический классификатор, основанный на применении теоремы Байеса с предусловием независимости признаков. Первые упоминания практического применения теоремы Байеса относятся к 18 веку. \cite{Stigler:1983}
	
	\subsection{Байесовская сеть}
	\label{subsec:bayes_net}
		Байесовская сеть является графом, в вершинах которого находятся переменные, а дуги выражают зависимость между этими переменными. Таким образом, сеть выражает отношения между переменными, что позволяет строить классификаторы на ее основе. Данный подход к классификации был формализован Перлом (Pearl) в \cite{Pearl:1985}. В данной работе используется реализация Байесовской сети из фреймворка Weka \cite[с.~111-123]{WekaManual}.