\chapter*{Реферат}
\addcontentsline{toc}{chapter}{Реферат}

Задача ранжирования одна из наиболее активно развиваемых областей обучения предпочтениям. В этой работе представлена реализация алгоритма выявления предпочтений при прочих равных. Этот алгоритм работает с попарными предпочтениями, а значит для его использования не требуется выявления функции полезности. В работе представлено эмпирическое исследование реализации, основанной на формальном определении алгоритма. Количественные показатели результатов экспериментов, проведенных над алгоритмом, сравниваются с другими методами машинного обучения. Так же в работе представлена модификация, расширяющая область применения алгоритма. Проведенные эксперименты показали, что алгоритм адекватно работает на реальных наборах данных.

\textbf{Ключевые слова}: обучение предпочтениям, машинное обучение, предпочтения при прочих равных, обучение ранжированию, попарные предпочтения 

\vspace{2em}

Работа состоит из 32 страниц, 3 глав, содержит 2 таблицы, 1 иллюстрацию и 32 источника. 

\newpage


Ranking problem is one of the most extensively studied field of preference learning. In this paper an implementation of the ceteris paribus preference elicitation algorithm is presented. This algorithm works with binary relations and supports purely relative data that has no utility function. Based on the formal description of the algorithm, we conduct an empirical study of its implementation. The quantitative measurement of the implementation is compared with various machine learning techniques. In addition, we propose a modification of the algorithm, which expand its scope. Results of conducted experiments proves the ability of the algorithm to work with real-world datasets.

\textbf{Keywords}: preference learning, machine learning, ceteris paribus preferences, learning to rank, pair preferences