\chapter{Модификация алгоритма}
\label{chapter:modification}

Алгоритм выявления предпочтений \enquote{при прочих равных} работает лишь с бинарными признаками. Однако, наборы данных, рассматриваемые в данной работе, содержат в том числе многозначные и числовые признаки. В этом разделе представлена модификация Алгоритма, которая умеет работать с небинарными признаками. В главе~\ref{chapter:experiments} представлены результаты экспериментов не только над базовой версией Алгоритма, но и над модифицированной.

В основе модификации лежит идея изменения $F$ (см. опр.~\ref{def:ceteris_paribus}): вместо множества признаков используется множество предикатов. При этом меняется семантика Алгоритма. В статье \cite{Obiedkov:2013}, где он был представлен, использовались только бинарные признаки (например, экстерьер мог быть только красным или белым, см. рис.~\ref{fig:pcxt}). Таким образом, из появления в $F$ одного из признаков группы (например, группы признаков ``экстерьер'') следовало наличие или отсутствие второго признака. При использовании предикатов множество $F$ представляет набор утверждений вида ``левая и правая часть равны по признаку $c$'', где $c$ – некая категория признаков. Переход от множества признаков ко множеству предикатов позволяет строить более сложные утверждения (например, рассматривать неравенства), а так же работать с многозначными и числовыми признаками в дополнение к бинарным. Определение \ref{def:multivalued_context}\cite{Ganter:1997} является формальным описанием \emph{многозначного контекста}, который может содержать небинарные признаки.

\begin{definition}
\label{def:multivalued_context}
	\emph{Многозначный (формальный) контекст} – это кортеж $\context^* = (G, M^*, V, J)$, где $G$ называется множеством объектов, $M^*$ – множество признаков, $V$ – множество \emph{значений}, а J является тернарным отношением ${I \subseteq G \times M^* \times V}$, содержащим тройки $(g\:m\:v)$. $(g\:m\:v)$ интерпретируется как ``значение признака $m$ объекта $g$ равно $v$''. 
\end{definition}

Контекст предпочтений, основанный на $\context^*$ (\emph{многозначный контекст предпочтений}), будем обозначать $\PP^*$. Аналогично опр.~\ref{def:preference_context}, $\PP^* = (G, M^*, V, J, \leq)$.

Обозначим $\mathbb{A} = \{A_1, A_2, \dots, A_n\}$ множество всех категориальных многозначных признаков, а $\mathbb{C} = \{C_1, C_2, \dots, C_m\}$ множество всех числовых признаков. Тогда $M^* = \mathbb{A} \cup \mathbb{C}$.

Нами вводятся 3 типа предикатов, а так же множества, состоящие из представленных предикатов:
\begin{subequations} %TODO: merge condition
	\begin{gather} 
	\label{eq:eq_predicate}
	p^=_m(v^*_i, v^*_j) := (v^*_i = v^*_j) \\ 
	% 
	\label{eq:lt_predicate}
	p^<_m(v_i, v_j) := (v_i < v_j) \\ 
	%
	\label{eq:gt_predicate}
	p^>_m(v_i, v_j) := (v_i > v_j)
	\end{gather}
\end{subequations}
где $v^*_i$ и $v^*_j$ в \eqref{eq:eq_predicate} – значения некого признака m, \\
$v_i$ и $v_j$ в \eqref{eq:lt_predicate} и \eqref{eq:gt_predicate} – некоторые значения числового признака $m$. \\
\begin{subequations} 
	\begin{gather}
	\label{eq:all_predicates_m}
	P_m = \{p^=_m, p^<_m, p^>_m\} \\
	%
	\label{eq:all_predicates}
	\mathbb{P} = \bigcup_{m \in M^*}P_m
	\end{gather}
\end{subequations}
где $P_m$ содержит все предикаты для признака $m$, \\
$\mathbb{P}$ содержит все предикаты для всех признаков.\\
В дальнейшем введенные предикаты используются для попарного выявления отношений между признаками двух объектов.\\
Обозначим через $\mathtt{v}_m(g)$ функцию, которая показывает значение признака $m$ объекта $g$:
\begin{equation}
\mathtt{v}_m(g) = v\;|\;(g\:m\:v) \in J \quad \text{для данного $\context^*$}
\end{equation} \\
Для удобства определим предикаты для объектов:
\begin{equation}
\label{eq:obj_predicate}
p^{\{<,=,>\}}_m(x, y) := p^{\{<,=,>\}}_m(\mathtt{v}_m(x), \mathtt{v}_m(y))
\end{equation}
где $x$ и $y$ – некоторые объекты. \\
Например, существуют два объекта $x$ и $y$, которые представляют автомобили: седан с ручной трансмиссией и объемом двигателя 3.5L; и SUV с ручной трансмиссией и объемом двигателя 5.5L. Тогда для пары объектов $(x, y)$ будут выполняться предикаты $p^=_{\text{тран.}}(x, y)$ и  $p^<_{\text{двиг.}}(x, y)$, но не будут выполняться  $p^>_{\text{двиг.}}(x, y)$, $p^=_{\text{кузов}}(x, y)$.

Алгоритмы \ref{algo:prediction_multi} и \ref{algo:check_multi} являются модификациями алгоритмов \ref{algo:prediction} и \ref{algo:check}, соответственно. В отличие от своих базовых версий, данные алгоритмы работают с многозначным контекстом предпочтений. А значит, меняется семантика множества $F$: в отличие от Алг.~\ref{algo:prediction}, в котором $F$ содержало признаки, по которым совпадают $A$ и $B$, а также $g$ и $h$, $F$ в Алг.~\ref{algo:prediction_multi} содержит предикаты, которые выполняются и для объектов $A'$ и $B'$ (объекты представлены признаками $A$ и $B$), и для $g$ и $h$.

\begin{algorithm}
	\caption{\algname{Предсказание предпочтения}$(A, B, \PP^*)$ (Основан на Алг.~\ref{algo:prediction})}
	\label{algo:prediction_multi}
	\begin{algorithmic}[1]
		\REQUIRE Содержания объектов $A, B \subseteq M^* \times V$ и контекст предпочтений $\PP^* = (G, M^*, V, J, \leq)$.
		\ENSURE \TRUE, если $\PP^*$ поддерживает $\DEF$ для некоторого $D \subseteq A, E \subseteq B,$ и $F \subseteq \mathbb{P}$ таких, что $F \cap A = F \cap B$; \FALSE, иначе.
		\item[]
		\FORALL{$g \in G$}
		\STATE $D := A \cap g'$
		\FORALL{$h \in G \setminus \{g\}$ таких, что $g \leq h$}
		\STATE $E := B \cap h'$
		\STATE $F_{ab} := \{p\;|\;(p \in \mathbb{P}) \wedge p\,(A', B')\}$
		\STATE $F_{gh} := \{p\;|\;(p \in \mathbb{P}) \wedge p\,(g, h)\}$
		\STATE $F := F_{ab} \cap F_{gh}$
		\IF{$\PP \models \DEF$}
		\RETURN \TRUE
		\ENDIF
		\ENDFOR
		\ENDFOR
		\RETURN \FALSE
	\end{algorithmic}
\end{algorithm}

\begin{algorithm}
	\caption{\algname{Проверить предпочтение}$(\DEF, \PP^*)$ (Основан на Алг.~\ref{algo:check})}
	\label{algo:check_multi}
	\begin{algorithmic}[1]
		\REQUIRE Предпочтение $\DEF$ над $M$ и контекст предпочтений $\PP^* = (G, M^*, V, J, \leq)$
		\ENSURE \TRUE, если $\PP^* \models \DEF$; \FALSE, иначе.
		\item[]
		\STATE $X := \bigcap_{m \in D}m'$
		\STATE $Y := \bigcap_{m \in E}m'$
		\FORALL{$g \in X$}
		\FORALL{$h \in Y$}
		\IF {$g \not\leq h$ \AND $p\,(g, h)\: \forall p \in F$}
		\RETURN \FALSE
		\ENDIF
		\ENDFOR
		\ENDFOR
		\RETURN \TRUE
	\end{algorithmic}
\end{algorithm}
