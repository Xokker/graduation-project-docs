\chapter{Модификация алгоритма}
\label{chapter:modification}

Заменяем $F$: вместо множества признаков становится множество предикатов. При этом меняется семантика алгоритма. В статье \cite{Obiedkov:2013}, где был представлен алгоритм выявления предпочтений \enquote{при прочих равных}, использовались только бинарные признаки (например, экстерьер мог быть только красным или белым, см. рис.~\ref{fig:pcxt}). Таким образом, из появления в $F$ одного из признаков группы (например, группы признаков ``экстерьер'') следовало наличие или отсутствие второго признака. При использовании предикатов множество $F$ представляет набор утверждений вида ``левая и правая часть равны по признаку $c$'', где $c$ – некая категория признаков. Переход от множества признаков ко множеству предикатов позволяет строить более сложные утверждения (например, рассматривать неравенства).

