\chapter{Модификация алгоритма}
\label{chapter:modification}



Заменяем $F$: вместо множества признаков становится множество предикатов. При этом меняется семантика алгоритма. В статье \cite{Obiedkov:2013}, где был представлен алгоритм выявления предпочтений \enquote{при прочих равных}, использовались только бинарные признаки (например, экстерьер мог быть только красным или белым, см. рис.~\ref{fig:pcxt}). Таким образом, из появления в $F$ одного из признаков группы (например, группы признаков ``экстерьер'') следовало наличие или отсутствие второго признака. При использовании предикатов множество $F$ представляет набор утверждений вида ``левая и правая часть равны по признаку $c$'', где $c$ – некая категория признаков. Переход от множества признаков ко множеству предикатов позволяет строить более сложные утверждения (например, рассматривать неравенства), а так же работать с многозначными и числовыми признаками в дополнение к бинарным. Определение \ref{def:multivalued_context}\cite{Ganter:1997} дает формальное описания \emph{многозначного контекста}, который может содержать небинарные (многозначные и числовые) признаки.

\begin{definition}
\label{def:multivalued_context}
	\emph{Многозначный (формальный) контекст} – это кортеж $\context^* = (G, M^*, V, J)$, где $G$ называется множеством объектов, $M^*$ – множество признаков, $V$ – множество \emph{значений}, а J является тернарным отношением ${I \subseteq G \times M^* \times V}$, содержащим тройки $(g\:m\:v)$. $(g\:m\:v)$ интерпретируется как ``значение признака $m$ объекта $g$ равно $v$''. 
\end{definition}

Контекст предпочтений, основанный на $\context^*$, будем обозначать $\PP^*$. Аналогично опр.~\ref{def:preference_context}, $\PP = (G, M^*, V, J, \leq)$.

Обозначим $\mathbb{A} = \{A_1, A_2, \dots, A_n\}$ множество всех категориальных многозначных признаков, а $\mathbb{C} = \{C_1, C_2, \dots, C_m\}$ множество всех числовых признаков. Тогда $M^* = \mathbb{A} \cup \mathbb{C}$.

Нами вводятся 3 типа предикатов, так же множество, состоящее из всех таких предикатов:
\begin{subequations} %TODO: merge condition
	\begin{gather} 
	\label{eq:eq_predicate}
	p^=_m(v^*_i, v^*_j) := (v^*_i = v^*_j) \\ 
	% 
	\label{eq:lt_predicate}
	p^<_m(v_i, v_j) := (v_i < v_j) \\ 
	%
	\label{eq:gt_predicate}
	p^>_m(v_i, v_j) := (v_i > v_j)
	\end{gather}
\end{subequations}
где $v^*_i$ и $v^*_j$ в \eqref{eq:eq_predicate} – значения некого признака m, \\
$v_i$ и $v_j$ в \eqref{eq:lt_predicate} и \eqref{eq:gt_predicate} – некоторые значения числового признака $m$. \\
\begin{subequations} 
	\begin{gather}
	\label{eq:all_predicates_m}
	P_m = \{p^=_m, p^<_m, p^>_m\} \\
	%
	\label{eq:all_predicates}
	\mathbb{P} = \{P_m\;|\;m \in M^*\}
	\end{gather}
\end{subequations}
где $P_m$ содержит все предикаты для признака $m$, \\
$\mathbb{P}$ содержит $P_m$ для всех признаков.\\
В дальнейшем введенные предикаты используются для попарного выявления отношений между признаками двух объектов.\\
Обозначим через $\mathtt{v}_m(g)$ функцию, которая показывает значение признака $m$ объекта $g$. Формально:
\begin{equation}
\mathtt{v}_m(g) = v\;|\;(g\:m\:v) \in J \quad \text{для данного $\context^*$}
\end{equation} \\
Для удобства определим предикаты для объектов:
\begin{equation}
\label{eq:obj_predicate}
p^{\{<,=,>\}}_m(x, y) := p^{\{<,=,>\}}_m(\mathtt{v}_m(x), \mathtt{v}_m(y))
\end{equation}
где $x$ и $y$ – некоторые объекты. \\
Например, есть два объекта $x$ и $y$, которые представляют автомобили: седан с ручной трансмиссией и объемом двигателя 3.5L; и SUV  ручной трансмиссией и объемом двигателя 5.5L. Тогда для этих объектов будут выполняться следующие равенства: $p^=_{\text{кузов}}(x, y) = 0$, $p^=_{\text{тран.}}(x, y) = 1$, $p^<_{\text{двиг.}}(x, y) = 1$, $p^>_{\text{двиг.}}(x, y) = 0$

\begin{algorithm}
	\caption{\algname{Предсказание предпочтения}$(A, B, \PP^*)$ \cite[Алг.~1]{Obiedkov:2013}}
	\label{algo:prediction_multi}
	\begin{algorithmic}[1]
		\REQUIRE Содержания объектов $A, B \subseteq M^*$ и контекст предпочтений $\PP^* = (G, M^*, V, J, \leq)$.
		\ENSURE \TRUE, если $\PP^*$ поддерживает $\DEF$ для некоторого $D \subseteq A, E \subseteq B,$ и $F \subseteq \mathbb{P}$ таких, что $F \cap A = F \cap B$; \FALSE, иначе.
		\item[]
		\FORALL{$g \in G$}
		\STATE $D := A \cap g'$
		\FORALL{$h \in G \setminus \{g\}$ таких, что $g \leq h$}
		\STATE $E := B \cap h'$
		\STATE $F_{ab} := \{p\;|\;(p \in \mathbb{P}) \wedge p(\mathtt{v}_m(A), \mathtt{v}_m(B))\}$
		\STATE $F := (M \setminus (A \vartriangle B)) \cap (M \setminus (g' \vartriangle h'))$
		\IF{$\PP \models \DEF$}
		\RETURN \TRUE
		\ENDIF
		\ENDFOR
		\ENDFOR
		\RETURN \FALSE
	\end{algorithmic}
\end{algorithm}
