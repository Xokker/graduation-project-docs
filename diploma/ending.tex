\chapter*{Заключение}
\addcontentsline{toc}{chapter}{Заключение}
\label{ch:ending}

Основной задачей данной работы было экспериментальное исследование алгоритма выявления предпочтений \enquote{при прочих равных}, которое заключается в апробации Алгоритма на реальных наборах данных, а так же в сравнении результатов экспериментов с результатами уже существующих методов. В ходе работы на языке Java был реализован Алгоритм и проведено исследование качества его работы на основе двух реальных наборов данных. Для оценки качества работы Алгоритма были проведены эксперименты над теми же наборами данных, но с использованием других алгоритмов: C4.5, наивный Байесовский классификатор и клссификатор, основанный на Байесовской сети.

В ходе работы разработаны формальные описания нескольких модификаций Алгоритма, а так же проведено эмпирическое исследование работы их реализаций. Часть модификаций являются простыми адаптациями, задачей которых было не меняя семантики оригинального Алгоритма повысить качественные показатели его работы при проверке на реальных данных. Модификация, основанная на предикатах, изменяет семантику Алгоритма, расширяя его возможности. 

Проведенные эксперименты показали, что алгоритм выявления предпочтений \enquote{при прочих равных} может быть успешно использован на реальных данных. В первой серии экспериментов показатели результатов работы простых модификаций оказались сопоставимы с широко используемыми методами машинного обучения, а модификация, основанная на предикатах, по показателям полноты превзошла C4.5 с числовыми признаками (\emph{C4.5 paired, num}), который показывал лучшие результаты в исследовании. Вторая серия экспериментов, которые проводились на наборе данных с суши, показала превосходство Алгоритма над рассматриваемыми методами машинного обучения. 
