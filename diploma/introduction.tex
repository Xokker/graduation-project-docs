\chapter*{Введение}
\addcontentsline{toc}{chapter}{Введение}

Обучение предпочтениям (preference learning) – одна из наиболее развиваемых областей машинного обучения. Главная задача алгоритмов выявления предпочтений – конструирование модели, основанной на тренировочном наборе данных, и предсказание отношений предпочтений в случае добавления новых объектов в модель.

В настоящее время исследователи наибольшее внимание уделяют задачи ранжирования. У обучения ранжированию существует большое количество приложений. Например, розничные сети могут предсказывать предпочтения клиента, основываясь на его демографических данных, таких как пол, возраст, семейное положение и т.п. Другим примером является биоинформатика, где для упорядочивания генов применяется ранжирование, основанное на филогенетических данных. \cite{Balasubramaniyan:2005} Поисковые движки используют алгоритмы ранжирования для упорядочивания поисковой выдачи. Более того, обучение ранжированию может быть использовано для упорядочивания самих алгоритмов ранжирования\cite{Brazdil:2003}.

Основная часть данной работы посвящена различным алгоритмам выявления предпочтений. Задача таких алгоритмов заключается в ранжировании множества объектов. Обычно алгоритмы выявления предпочтений являются одним из ключевых модулей рекомендательных систем. Такие системы пытаются \enquote{угадать} предпочтение пользователя, используя информацию о его предыдущих действиях и действиях пользователей, похожих на него. Примерами таких систем являются сервисы imhonet.ru\footnote{http://imhonet.ru/about/ – о проекте Имхонет}, kinopoisk.ru\footnote{http://www.kinopoisk.ru/docs/join/ - о возможностях Кинопоиска}, last.fm\footnote{http://www.lastfm.ru/about – о проекте last.fm} и многие другие.

Основная \emph{цель} данной работы – исследование алгоритма выявления предпочтений \enquote{при прочих равных} (далее - Алгоритм), который был описан в \cite{Obiedkov:2013}. Основные \emph{задачи}:
\begin{enumerate}[itemsep=-1mm]
	\item Реализация Алгоритма на одном из прикладных языков программирования
	\item Сравнение результатов работы Алгоритма с другими алгоритмами обучения предпочтений и классическими методы машинного обучения
	\item Модификация Алгоритма, заключающаяся в расширении его области действия и повышения его производительности
\end{enumerate}

%\section*{Обоснование выбора алгоритма}
%TODO обоснование почему попарные предпочтения лучше семантического дифференциала

\section*{Структура работы}
В Главе~\ref{chapter:literature} представлен обзор литературы, посвященной методам выявления предпочтений. В Главе~\ref{chapter:theory} дан необходимый теоретический базис, описана классификация задач ранжирования, а так же представлен Алгоритм и показан пример его работы. Глава~\ref{chapter:experiments} посвящен проведенным над алгоритмами экспериментам. В ней описан метод сравнения алгоритмов, представлены наборы данных, на которых тестировались алгоритмы, и выделены полученные результаты. В главе~\ref{chapter:modification} описана модификация, которая позволяет расширить область применения Алгоритма. Последняя глава подводит итоги работы.