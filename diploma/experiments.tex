\chapter{Эксперименты}
\label{chapter:experiments}
В данной главе представлена экспериментальная часть работы. Сравнение результатов работы алгоритма выявления предпочтений \enquote{при прочих равных} с другими методами обучения предпочтениям является ключевой частью данного исследования.

Глава состоит из четырех секций. В первой описаны наборы данных, с помощью которых сравнивались алгоритмы. Во второй секции описаны сами алгоритмы, которые были рассмотрены в данной работе. В третьей секции описана методика сравнения алгоритмов. И в заключительной части представлены результаты экспериментов.

\section{Входные данные}
	В данной работе представлена
	апробация Алгоритма проводилась на трех наборах данных: тестовые данные с предпочтениями об автомобилях, реальный набор данных о пользовательских предпочтениях в автомобилях %TODO: переформулировать
	и реальный набор данных о предпочтениях в суши. Ниже подробно описан каждый из наборов данных.
	
	\subsection{Искусственный набор данных}
	\subsection{Реальный набор данных: автомобили}
	\subsection{Реальный набор данных: суши}
	
\section{Рассматриваемые алгоритмы}

\section{Методика}

$k$-fold cross-validation

\section{Результаты}