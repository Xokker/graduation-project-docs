\chapter{Эксперименты}
\label{chapter:experiments}
В данной главе представлена экспериментальная часть работы. Сравнение результатов работы алгоритма выявления предпочтений \enquote{при прочих равных} с другими методами обучения предпочтениям является ключевой частью данного исследования.

Глава состоит из четырех разделов. В первом описаны наборы данных, на которых сравнивались алгоритмы. Во втором разделе описаны сами алгоритмы, которые были рассмотрены в данной работе. В третьем описана методика сравнения алгоритмов. И в заключительной части представлены результаты экспериментов.

\section{Входные данные}
	В данной работе представлена
	апробация Алгоритма проводилась на трех наборах данных: тестовые данные с предпочтениями об автомобилях, реальный набор данных о пользовательских предпочтениях в автомобилях %TODO: переформулировать
	и реальный набор данных о предпочтениях в суши. Ниже подробно описан каждый из наборов данных.
	
	\subsection{Искусственный набор данных}
		Искусственный набор взят из \cite{Obiedkov:2013}. Этот набор данных состоит из семи объектов, пять из которых представлены на рис.~\ref{fig:pcxt}. Еще два объекта:
		$c_6={minivan, red, bright}$ и $c_7={SUV, red, bright}$. Эти данные не много могут сказать о качестве работы алгоритма, но их можно использовать для базовой проверки его возможностей.
	
	\subsection{Реальный набор данных: автомобили}
		Первый реальный набор данных собран Аббаснеджадом и др. для \cite{dataset:Abbasnejad:2013}. Для сбора пользовательских предпочтений авторы использовали краудсорсинговую платформу Amazon Mechanical Turk\footnote{mturk.com – информация о проекте Amazon Mechanical Turk}. Набор данных состоит из двух экспериментов\footnote{Данные размещены по адресу: http://users.cecs.anu.edu.au/~u4940058/CarPreferences.html}: в первом была собрана информация от 60 пользователей о 10 машинах, во втором – от 60 пользователей о 20 машинах. Причем во втором эксперименте информации о каждой из машин было больше. Далее подробно описана информация, представленная в наборе данных.
		
		\vspace{1em}
		
		\noindent Характеристики пользователей:
		\vspace{-0.7em}
		\begin{itemize}[itemsep=-1.5mm]
			\item ID: уникальный идентификатор пользователя
			\item Образование: отсутствие ответа (0), старшая школа (1), бакалавриат (2), PhD (3)
			\item Возраст: отсутствие ответа (0), меньше 25 (1), от 25 до 30 (2), от 30 до 35 (3), больше 40 (4)
			\item Пол: отсутствие ответа (0), мужской (1), женский (2)
			\item Регион: отсутствие ответа (0), юг (1), запад (2), северо-восток (3), средний запад (4)
			\item Количество правильных ответов на контрольные вопросы: 3, 4 или 5
		\end{itemize}
		Данные собирались среди американской аудитории. Каждому пользователю было задано по пять контрольных вопросов, каждый из которых является одним из пользовательских предпочтений в обратном порядке (например, если пользователь указал, что $o_1 < o_5$, то контрольный вопрос имеет вид $o_5 < o_1?$, где $o_1$ и $o_5$ – какие-то объекты). На основе количества правильных ответов на контрольные вопросы можно судить о согласованности пользовательских предпочтений.
		
		\vspace{1em}
		
		\noindent Признаки машин:
		\vspace{-0.7em}
		\begin{itemize}[itemsep=-1.5mm]
			\item Тип кузова: седан (1), SUV\footnote{SUV (англ. Sport Utility Vehicle или Suburban Utility Vehicle) – подобие внедорожника} (2), хэтчбек (3)
			\item Коробка передач: ручная (1), автоматическая (2)
			\item Объем двигателя: 2.5L, 3.5L, 4.5L, 5.5L, 6.2L
			\item Тип топлива: гибрид (1), не гибрид (2)
			\item Количество ведущих колес: все колеса ведущие (AWD, 4x4) (1), передние колеса ведущие (FWD) (2)
		\end{itemize} 
		В первом эксперименте не были представлены хэтчбеки, а так же не использовалась информация о количестве ведущих колес.
	
	\subsection{Реальный набор данных: суши}
	
\section{Рассматриваемые алгоритмы}

\section{Методика}

$k$-fold cross-validation

\section{Результаты}