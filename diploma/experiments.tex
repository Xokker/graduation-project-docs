\chapter{Эксперименты}
\label{chapter:experiments}
В данной главе представлена экспериментальная часть работы. Сравнение результатов работы алгоритма выявления предпочтений \enquote{при прочих равных} с другими методами обучения предпочтениям является ключевой частью данного исследования.

Глава состоит из четырех разделов. В первом описаны наборы данных, на которых сравнивались алгоритмы. Во втором разделе описаны сами алгоритмы, которые были рассмотрены в данной работе. В третьем описана методика сравнения алгоритмов. И в заключительной части представлены результаты экспериментов.

\section{Входные данные}
	В данной работе представлена
	апробация Алгоритма проводилась на трех наборах данных: тестовые данные с предпочтениями об автомобилях, реальный набор данных о пользовательских предпочтениях в автомобилях %TODO: переформулировать
	и реальный набор данных о предпочтениях в суши. Ниже подробно описан каждый из наборов данных.
	
	\subsection{Искусственный набор данных}
		Искусственный набор взят из \cite{Obiedkov:2013}. Этот набор данных состоит из семи объектов, пять из которых представлены на рис.~\ref{fig:pcxt}. Еще два объекта:
		$c_6={minivan, red, bright}$ и $c_7={SUV, red, bright}$. Эти данные не много могут сказать о качестве работы алгоритма, но их можно использовать для базовой проверки его возможностей.
	
	\subsection{Реальный набор данных: автомобили}
		Первый реальный набор данных собран Аббаснеджадом и др. для \cite{dataset:Abbasnejad:2013}. Для сбора пользовательских предпочтений авторы использовали краудсорсинговую платформу Amazon Mechanical Turk\footnote{mturk.com – информация о проекте Amazon Mechanical Turk}. Набор данных состоит из двух экспериментов\footnote{Данные размещены по адресу: http://users.cecs.anu.edu.au/~u4940058/CarPreferences.html}: в первом была собрана информация от 60 пользователей о 10 машинах, во втором – от 60 пользователей о 20 машинах (далее – CD1 и CD2 обозначают первый и второй эксперимент, соответственно). Причем в CD2 информации о каждой из машин было больше. Далее подробно описана информация, представленная в наборе данных.
		
		\vspace{1em}
		
		\noindent Характеристики пользователей:
		\vspace{-0.7em}
		\begin{itemize}[itemsep=-1.5mm]
			\item ID: уникальный идентификатор пользователя
			\item Образование: отсутствие ответа (0), старшая школа (1), бакалавриат (2), PhD (3)
			\item Возраст: отсутствие ответа (0), меньше 25 (1), от 25 до 30 (2), от 30 до 35 (3), больше 40 (4)
			\item Пол: отсутствие ответа (0), мужской (1), женский (2)
			\item Регион: отсутствие ответа (0), юг (1), запад (2), северо-восток (3), средний запад (4)
			\item Количество правильных ответов на контрольные вопросы: 3, 4 или 5
		\end{itemize}
		Данные собирались среди американской аудитории. Каждому пользователю было задано по пять контрольных вопросов, каждый из которых является одним из пользовательских предпочтений в обратном порядке (например, если пользователь указал, что $o_1 < o_5$, то контрольный вопрос имеет вид $o_5 < o_1?$, где $o_1$ и $o_5$ – какие-то объекты). На основе количества правильных ответов на контрольные вопросы можно судить о согласованности пользовательских предпочтений.
		
		\vspace{1em}
		
		\noindent Признаки машин:
		\vspace{-0.7em}
		\begin{enumerate}[itemsep=-1.5mm]
			\item Тип кузова: седан (1), SUV\footnote{SUV (англ. Sport Utility Vehicle или Suburban Utility Vehicle) – подобие внедорожника} (2), хэтчбек (3)
			\item Коробка передач: ручная (1), автоматическая (2)
			\item Объем двигателя: 2.5L, 3.5L, 4.5L, 5.5L, 6.2L
			\item Тип топлива: гибрид (1), не гибрид (2)
			\item Количество ведущих колес: все колеса ведущие (AWD, 4x4) (1), передние колеса ведущие (FWD) (2)
		\end{enumerate} 
		В первом эксперименте не были представлены хэтчбеки, а так же не использовалась информация о количестве ведущих колес.
	
	\subsection{Реальный набор данных: суши}
		Набор данных собран Камишимой и Акахо и использован ими в \cite{Kamishima:2003}, \cite{Kamishima:2006} и других работах. Авторы опрашивали 5000 человек об их предпочтениях в суши (всего 100 наименований). При сборе данных авторы просили пользователей заполнить несколько опросников, в результате чего были сформированы 3 набора данных\footnote{Набор данных о предпочтениях суши доступен по адресу: http://www.kamishima.net/sushi/}:
		\begin{enumerate}[itemsep=-1.5mm]
			\item Выбрав 10 наиболее популярных суши, авторы попросили каждого из участников проранжировать их. В результате получен список из 5000 ранжирований. (в дальнейшем этот набор данных обозначается как SDa)
			\item Выбирая 10 случайных суши из 100 (выбор не равновероятностный, он основан на популярности суши), авторы предлагали каждому из пользователей их проранжировать. (в дальнейшем этот набор данных обозначается как SDb)
			\item Используя тот же набор суши (10 из 100), участники опроса должны были поставить каждой из альтернатив оценку от 0 до 4, включительно. В работе эти данные не используются.
		\end{enumerate}
		
		В связи со спецификой японской культуры еды, авторы подробно указывают место жительства каждого из участников, а так же место его жительства до 15 лет. На основе этой информации для выявления предпочтений можно использовать методы коллаборативной фильтрации\cite{Ricci:2011}, одну из вариаций которой авторы используют в \cite{Kamishima:2003}.
		
		\vspace{1em}
		
		\noindent Характеристики пользователей:
		\vspace{-0.7em}
		\begin{enumerate}[itemsep=-1.5mm]
			\item ID: уникальный идентификатор пользователя
			\item Пол: мужской (0), женский (1)
			\item Возраст: от 15 до 19 (0), от 20 до 29 (1), от 30 до 39 (2), от 40 до 49 (3), от 50 до 59 (4), больше 60 (5)
			\item Количество времени (секунд), которое ушло у участника на заполнение анкеты
			\itemrange{6} Признаки, связанные с местом проживания человека
		\end{enumerate}
		
		\noindent Признаки суши:
		\vspace{-0.7em}
		\begin{enumerate}[itemsep=-1.5mm]
			\item ID: уникальный идентификатор суши
			\item Название
			\item Стиль: маки (0), другое (1)
			\item Группа: морская еда (0, соответствует подгруппам 0--8), другое (1)
			\item Подгруппа: синекожая рыба (0), красная рыба (1), \dots, овощи (11) 
			\item Жирность: диапазон [0-4], где 0 -- наибольшая жирность
			\item Частота употребления этого вида суши участниками опроса: диапазон [0-3], где 3 -- наибольшая частота
			\item Цена (нормализованная)
			\item Частота продажи суши: диапазон [0-1], где 1 -- наибольшая частота
		\end{enumerate}
		
\section{Рассматриваемые алгоритмы}

\section{Проведение экспериментов}

$k$-fold cross-validation

%\section{Результаты}