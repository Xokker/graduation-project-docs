\chapter{Эксперименты}
\label{chapter:experiments}
В данной главе представлена экспериментальная часть работы. Сравнение результатов работы алгоритма выявления предпочтений \enquote{при прочих равных} с другими методами обучения предпочтениям является ключевой частью данного исследования.

Глава состоит из четырех разделов. В первом описаны наборы данных, на которых сравнивались алгоритмы. Во втором разделе описаны сами алгоритмы, которые были рассмотрены в данной работе. В третьем описана методика сравнения алгоритмов. И в заключительной части представлены результаты экспериментов.

\section{Входные данные}
	В данной работе представлена
	апробация Алгоритма проводилась на трех наборах данных: тестовые данные с предпочтениями об автомобилях, реальный набор данных о пользовательских предпочтениях в автомобилях %TODO: переформулировать
	и реальный набор данных о предпочтениях в суши. Ниже подробно описан каждый из наборов данных.
	
	\subsection{Искусственный набор данных}
		Искусственный набор взят из \cite{Obiedkov:2013}. Этот набор данных состоит из семи объектов, пять из которых представлены на рис.~\ref{fig:pcxt}. Еще два объекта:
		$c_6={minivan, red, bright}$ и $c_7={SUV, red, bright}$. Эти данные не много могут сказать о качестве работы алгоритма, но их можно использовать для базовой проверки его возможностей.
	
	\subsection{Реальный набор данных: автомобили}
		
	
	\subsection{Реальный набор данных: суши}
	
\section{Рассматриваемые алгоритмы}

\section{Методика}

$k$-fold cross-validation

\section{Результаты}